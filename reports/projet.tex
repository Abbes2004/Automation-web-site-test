\documentclass[12pt,a4paper]{report}

% =========================
% PACKAGES
% =========================
\usepackage[utf8]{inputenc}
\usepackage[T1]{fontenc}
\usepackage[french]{babel}
\usepackage{graphicx}
\usepackage{float}
\usepackage{longtable}
\usepackage{array}
\usepackage{geometry}
\usepackage{hyperref}
\usepackage{setspace}
\usepackage{titlesec}
\usepackage{xcolor}
\usepackage{booktabs}

\geometry{margin=2.5cm}
\onehalfspacing

% =========================
% TITRES
% =========================
\titleformat{\chapter}{\bfseries\Huge}{\thechapter}{20pt}{}
\titleformat{\section}{\bfseries\Large}{\thesection}{15pt}{}
\titleformat{\subsection}{\bfseries\large}{\thesubsection}{10pt}{}

\begin{document}

% =========================
% PAGE DE GARDE
% =========================
\begin{titlepage}
    \centering
    \vspace{3cm}
    {\Huge \textbf{Projet ISTQB}}\\[1.5cm]
    {\Large Tests logiciels automatisés d’une application SaaS}\\[2cm]

    \textbf{Auteurs :}\\
    Amine Abbes\\
    Med Bechir Torki\\[2cm]

    \textbf{Application testée :}\\
    Plateforme de réservation en ligne \textbf{Trafft}\\[2cm]

    \vfill
    Année universitaire 2024–2025
\end{titlepage}

\tableofcontents
\listoftables
\begin{table}[H]
\centering
\caption{Classification des types de tests}
\begin{tabular}{|l|p{10cm}|}
\hline
\textbf{Type de test} & \textbf{Description} \\ \hline
Tests fonctionnels & Validation des fonctionnalités principales \\ \hline
Tests négatifs & Vérification du rejet des entrées invalides \\ \hline
Tests BVA & Analyse des valeurs limites \\ \hline
Tests basés sur les états & Vérification des transitions de réservation \\ \hline
Tests de performance & Mesure des temps de réponse \\ \hline
Tests de stress & Comportement sous charge élevée \\ \hline
Tests cross-browser & Compatibilité multi-navigateurs \\ \hline
\end{tabular}
\end{table}
	\begin{table}[H]
\centering
\caption{Environnement matériel et logiciel de test}
\begin{tabular}{|l|l|}
\hline
\textbf{Élément} & \textbf{Valeur} \\ \hline
Système d’exploitation & Windows 10 64 bits \\ \hline
Navigateur principal & Google Chrome \\ \hline
Navigateurs secondaires & Firefox, Microsoft Edge \\ \hline
Langage & Python 3.11 \\ \hline
Framework de test & PyTest \\ \hline
Outil d’automatisation & Selenium WebDriver \\ \hline
IDE & PyCharm \\ \hline
\end{tabular}
\end{table}
\begin{table}[H]
\centering
\caption{Résultats globaux des tests}
\begin{tabular}{|l|c|}
\hline
\textbf{Indicateur} & \textbf{Valeur} \\ \hline
Nombre total de cas de test & 22 \\ \hline
Cas de test réussis & 18 \\ \hline
Cas de test échoués & 4 \\ \hline
Taux de réussite & 81.8\% \\ \hline
Durée totale d’exécution & 50 minutes \\ \hline
\end{tabular}
\end{table}





\listoffigures
\begin{figure}[H]
\centering
\fbox{\includegraphics[width=0.9\textwidth]{screenshots/structure du projet.png}}
\caption{Architecture générale du projet de test automatisé}
\end{figure}

\begin{figure}[H]
\centering
\fbox{\includegraphics[width=0.9\textwidth]{screenshots/test work flow.png}}
\caption{Cycle de vie du test selon l’ISTQB}
\end{figure}

\begin{figure}[H]
\centering
\fbox{\includegraphics[width=0.85\textwidth]{screenshots/passed TC.png}}
\caption{Exécution réussie d’un cas de test automatisé}
\end{figure}





\newpage

% =========================
\chapter{Introduction}
% =========================
Le test logiciel constitue une activité essentielle dans le cycle de vie du
développement des systèmes informatiques modernes. Avec la généralisation des
applications web et des plateformes SaaS, la qualité logicielle est devenue un
facteur déterminant de la satisfaction utilisateur et de la pérennité des
entreprises numériques.

Dans ce contexte, la certification ISTQB fournit un cadre méthodologique
internationalement reconnu pour structurer les activités de test, depuis
l’analyse des exigences jusqu’à la gestion des anomalies. Le présent projet
s’inscrit dans cette démarche et vise à appliquer concrètement les concepts
théoriques à travers un cas réel.

Ce travail porte sur l’automatisation de tests logiciels pour une application
web de réservation de services. L’automatisation a été choisie afin de garantir
la répétabilité des tests, d’améliorer la couverture fonctionnelle et de réduire
les erreurs humaines.

Ce rapport présente successivement l’application testée, la méthodologie
adoptée, les techniques de test utilisées, les résultats obtenus ainsi que les
anomalies détectées, avant de conclure sur les limites et les perspectives
d’amélioration.

% =========================
\chapter{Présentation de l’application Trafft}
% =========================
Trafft est une plateforme SaaS dédiée à la réservation de services en ligne.
Elle vise à digitaliser l’ensemble du parcours client, depuis la découverte
des services jusqu’à la confirmation et la gestion des rendez-vous.

\section{Fonctionnalités principales}
L’application permet aux utilisateurs de consulter un catalogue de services,
de sélectionner un prestataire (barber), de choisir un lieu (location) et de
réserver un créneau horaire via un calendrier interactif. Les utilisateurs
peuvent effectuer ces actions en tant qu’invités ou via un compte authentifié.

\section{Parcours utilisateur}
Le parcours utilisateur repose sur une succession d’étapes critiques :
sélection du service, choix du professionnel, sélection de la date et de
l’heure, saisie des informations personnelles et confirmation finale. Toute
erreur à l’une de ces étapes peut entraîner une perte de réservation ou une
insatisfaction client.

\section{Intérêt pour le test logiciel}
Du point de vue de l’ingénieur QA, Trafft présente plusieurs défis :
\begin{itemize}
    \item Gestion de formulaires complexes,
    \item Transitions d’états sensibles (annulation, reprogrammation),
    \item Problèmes de concurrence lors de réservations simultanées,
    \item Validation des données utilisateur.
\end{itemize}

% =========================
\chapter{Objectifs et périmètre des tests}
% =========================
L’objectif principal de ce projet est de vérifier la conformité fonctionnelle
et la robustesse de l’application Trafft face à différents scénarios
d’utilisation.

\section{Objectifs}
\begin{itemize}
    \item Détecter les défauts fonctionnels critiques,
    \item Valider les règles métier liées aux réservations,
    \item Évaluer le comportement de l’application sous charge,
    \item Vérifier la compatibilité multi-navigateurs.
\end{itemize}

\section{Périmètre}
Les tests couvrent les fonctionnalités accessibles côté client. Les aspects
internes tels que les API backend ou la base de données ne sont pas directement
testés dans ce projet.

% =========================
\chapter{Méthodologie de test ISTQB}
% =========================
La méthodologie adoptée suit le cycle de vie des tests défini par l’ISTQB.

\section{Analyse des exigences}
Les exigences ont été déduites à partir du comportement observable de
l’application et des règles métier implicites.

\section{Conception des tests}
Les cas de test ont été conçus en utilisant différentes techniques ISTQB afin
d’assurer une couverture maximale.

\section{Implémentation et exécution}
Les tests ont été automatisés à l’aide de Selenium WebDriver et du framework
PyTest, en appliquant le modèle Page Object Model pour améliorer la
maintenabilité.

% =========================
\chapter{Techniques de test utilisées}
% =========================
\section{Tests fonctionnels}
Ils permettent de vérifier que les fonctionnalités principales répondent aux
exigences.

\section{Tests négatifs}
Ces tests visent à s’assurer que l’application réagit correctement aux entrées
invalides.

\section{Analyse des valeurs limites}
La technique BVA a été utilisée pour tester les champs sensibles tels que le
nom, le téléphone et l’email.

\section{Tests basés sur les états}
Les transitions entre les différents états des réservations ont été vérifiées
afin d’éviter les incohérences.

\section{Tests de performance et de stress}
Ces tests évaluent la capacité du système à gérer plusieurs requêtes
simultanées et à maintenir des temps de réponse acceptables.

% =========================
\chapter{Architecture du projet de test}
% =========================
Le projet est structuré selon une architecture modulaire respectant les bonnes
pratiques de l’automatisation des tests.

\section{Page Object Model}
Chaque page de l’application est représentée par une classe dédiée, ce qui
permet de centraliser les localisateurs et les actions.

\section{Organisation des suites de tests}
Les tests sont regroupés par catégorie : fonctionnels, performance,
cross-browser et stress.

% =========================
\chapter{Présentation détaillée des cas de test}
% =========================
Les cas de test suivants ont été conçus et exécutés :

\begin{longtable}{|c|p{3cm}|p{7cm}|p{3cm}|}
\hline
\textbf{ID} & \textbf{Catégorie} & \textbf{Scénario} & \textbf{Statut attendu} \\ \hline
TC01 & Fonctionnel & Création d’un compte client & Succès \\ \hline
TC02 & Fonctionnel & Connexion valide & Succès \\ \hline
TC03 & Fonctionnel & Mot de passe erroné & Échec \\ \hline
TC04 & Fonctionnel & Email en majuscule & Échec \\ \hline
TC05 & Fonctionnel & Booking utilisateur connecté & Succès \\ \hline
TC06 & Fonctionnel & Booking invité & Succès \\ \hline
TC07 & BVA & Nom 255 caractères & Succès \\ \hline
TC08 & BVA & Nom 256 caractères & Échec \\ \hline
TC09 & BVA & Téléphone invalide & Échec \\ \hline
TC10 & BVA & Téléphone alphabétique & Échec \\ \hline
TC11 & BVA & Email invalide & Échec \\ \hline
TC12 & États & Annulation multiple & Succès \\ \hline
TC13 & États & Reprogrammation & Succès \\ \hline
TC14 & États & Déconnexion en cours & Succès \\ \hline
TC15 & Données & Vérification du prix & Succès \\ \hline
TC16 & Cross-browser & Chrome & Succès \\ \hline
TC17 & Cross-browser & Firefox & Succès \\ \hline
TC18 & Cross-browser & Edge & Succès \\ \hline
TC19 & Compatibilité & Mobile & Succès \\ \hline
TC20 & Compatibilité & Tablette & Succès \\ \hline
TC21 & Stress & Réservations simultanées & Échec \\ \hline
TC22 & Performance & Temps de chargement & Succès \\ \hline
\end{longtable}

% =========================
\chapter{Résultats d’exécution et analyse}
% =========================
L’exécution des tests a permis d’identifier plusieurs anomalies critiques,
notamment liées à la validation des données et à la gestion de la concurrence.

Un taux global de réussite élevé a été observé, mais les échecs détectés
présentent un impact métier important.

% =========================
\chapter{Gestion des anomalies}
% =========================
\\section{Bug TC04 – Email en majuscule}
\textbf{Description :} Le système accepte des emails en majuscules alors qu’un
rejet était attendu.

\textbf{Impact :} Risque de duplication de comptes et incohérence des données
d’authentification.

\begin{figure}[H]
\centering
\fbox{
\begin{minipage}{0.48\textwidth}
\centering
\includegraphics[width=\textwidth]{screenshots/TC04_1.png}
\caption*{Tentative de connexion avec email en majuscule}
\end{minipage}
\hfill
\begin{minipage}{0.48\textwidth}
\centering
\includegraphics[width=\textwidth]{screenshots/TC04_2.png}
\caption*{Accès accordé malgré la non-conformité}
\end{minipage}
}
\caption{Bug TC04 – Acceptation incorrecte des emails en majuscules}
\end{figure}


\section{Bug TC10 – Téléphone invalide}
\textbf{Description :} Le champ téléphone accepte des chaînes alphabétiques.

\textbf{Impact :} Données de contact non fiables et perte de communication avec
le client.

\begin{figure}[H]
\centering
\fbox{
\begin{minipage}{0.48\textwidth}
\centering
\includegraphics[width=\textwidth]{screenshots/TC10_1.png}
\caption*{Saisie d’un numéro alphabétique}
\end{minipage}
\hfill
\begin{minipage}{0.48\textwidth}
\centering
\includegraphics[width=\textwidth]{screenshots/TC10_2.png}
\caption*{Validation acceptée à tort}
\end{minipage}
}
\caption{Bug TC10 – Validation incorrecte du numéro de téléphone}
\end{figure}

\section{Bug TC12 – Confusion d’annulation}
\textbf{Description :} Le système confond les identifiants lors de multiples
annulations d’un même rendez-vous.

\textbf{Impact :} Incohérence des états de réservation et erreurs métier.

\begin{figure}[H]
\centering
\fbox{
\begin{minipage}{0.48\textwidth}
\centering
\includegraphics[width=\textwidth]{screenshots/TC12_1.png}
\caption*{Annulation du premier rendez-vous}
\end{minipage}
\hfill
\begin{minipage}{0.48\textwidth}
\centering
\includegraphics[width=\textwidth]{screenshots/TC12_2.png}
\caption*{Message d’erreur incorrect lors de la seconde annulation}
\end{minipage}
}
\caption{Bug TC12 – Confusion des états de réservation}
\end{figure}

\section{Bug TC21 – Concurrence}
\textbf{Description :} Le système accepte plusieurs réservations simultanées
pour le même créneau horaire.

\textbf{Impact :} Double booking critique impactant directement l’activité
métier.

\begin{figure}[H]
\centering
\fbox{\includegraphics[width=0.9\textwidth]{screenshots/TC21.png}}
\caption{Bug TC21 – Acceptation incorrecte de réservations concurrentes}
\end{figure}

% =========================
\chapter{Limites et perspectives}
% =========================
Les tests se limitent à la couche frontend. Des améliorations futures pourraient
inclure des tests API, une intégration CI/CD et l’utilisation de l’IA générative
pour la conception automatique des tests.

% =========================
\chapter{Conclusion}
% =========================
Ce projet a permis d’appliquer concrètement les concepts ISTQB à un cas réel.
L’automatisation a démontré son efficacité dans la détection d’anomalies
critiques et constitue un levier essentiel pour la qualité des applications
SaaS modernes.

\end{document}
